Sure, I'll create a LaTeX document that explains the fundamentals of Graph Theory using the provided template.

```latex
\documentclass{beamer}

% Theme setup
\usetheme{default}
\usepackage{graphicx}

% Color setup
\setbeamercolor{frametitle}{bg=blue,fg=black}  % Header background blue, title color black
\setbeamercolor{title in head/foot}{bg=blue,fg=white}

% Remove color from the footer
\setbeamercolor{footline}{bg=,fg=black}  % No background color for the footer, set text color to black

% Logo setup (right side of the header, with fine adjustments for vertical and horizontal positioning)
\setbeamertemplate{frametitle}{
	\vspace{0.2cm}
	\hbox{
		\begin{beamercolorbox}[wd=\paperwidth,ht=1cm,dp=0.3cm]{frametitle}
			\hspace{0.5cm} % Adjust horizontal space from the left
			\textbf{\textcolor{black}{\insertframetitle}} % Slide title in bold and black
			\hfill
			\raisebox{-0.1cm}{\includegraphics[height=0.9cm]{logo.png}}  % Adjust the raisebox value to move the logo up/down
			\hspace{0.5cm} % Adjust horizontal space from the right
		\end{beamercolorbox}
	}
}

% Footer setup for slide number with total page number (adjustable vertical and horizontal position)
\setbeamertemplate{footline}{
	\hbox{
		\hfill
		\raisebox{0.2cm}{ % Adjust vertical positioning (positive value moves it up, negative moves it down)
			\hspace*{11.8cm}  % Adjust horizontal positioning (move left or right)
			\textbf{\insertframenumber}/\inserttotalframenumber % Add page number and total number of slides
		}
		\hfill
	}
}

\begin{document}
	
	% Title Slide
	\begin{frame}
		\title{Fundamentals of Graph Theory}
		\subtitle{}
		\author{Ragupathi M}
		\institute{Ramanujan Computing Centre}
		\titlepage
	\end{frame}
	
	% Slide: Introduction to Graph Theory
	\begin{frame}
		\frametitle{Introduction to Graph Theory}
		\begin{itemize}
			\item Graph Theory is a field of mathematics about how points (nodes) are connected by lines (edges).
			\item Graphs are used to model pairwise relations between objects.
			\item Applications include computer science, biology, social sciences, and more.
		\end{itemize}
	\end{frame}
	
	% Slide: Basic Definitions
	\begin{frame}
		\frametitle{Basic Definitions}
		\begin{itemize}
			\item \textbf{Graph}: A set of vertices (nodes) and edges (lines connecting nodes).
			\item \textbf{Vertex (Node)}: A fundamental unit by which graphs are formed.
			\item \textbf{Edge (Connection/Link)}: A line connecting two vertices in a graph.
		\end{itemize}
		\begin{figure}
			\includegraphics[width=0.5\linewidth]{graph.png}
			\caption{An example of a graph.}
		\end{figure}
	\end{frame}
	
	% Slide: Types of Graphs
	\begin{frame}
		\frametitle{Types of Graphs}
		\begin{itemize}
			\item \textbf{Undirected Graph}: Edges have no direction.
			\item \textbf{Directed Graph (Digraph)}: Edges have a direction.
			\item \textbf{Weighted Graph}: Edges have weights (values).
		\end{itemize}
	\end{frame}
	
	% Slide: Graph Representations
	\begin{frame}
		\frametitle{Graph Representations}
		\begin{itemize}
			\item \textbf{Adjacency Matrix}: A square matrix used to represent a finite graph.
			\item \textbf{Adjacency List}: A collection of lists or arrays used to represent a graph.
		\end{itemize}
		\begin{figure}
			\includegraphics[width=0.6\linewidth]{adjacency_matrix.png}
			\caption{Example of an Adjacency Matrix.}
		\end{figure}
	\end{frame}
	
	% Slide: Fundamental Theorems
	\begin{frame}
		\frametitle{Fundamental Theorems}
		\begin{itemize}
			\item \textbf{Euler's Theorem}: Conditions for Eulerian paths and circuits.
			\item \textbf{Hamiltonian Path}: Visits each vertex exactly once.
			\item \textbf{Graph Coloring}: Coloring vertices so that no two adjacent vertices share the same color.
		\end{itemize}
	\end{frame}
	
	% Slide: Applications of Graph Theory
	\begin{frame}
		\frametitle{Applications of Graph Theory}
		\begin{itemize}
			\item \textbf{Computer Networks}: Modeling of LANs and WANs.
			\item \textbf{Social Networks}: Representation of relationships.
			\item \textbf{Biology}: Networks of neurons or protein interactions.
			\item \textbf{Transportation}: Route optimization.
		\end{itemize}
	\end{frame}
	
	% Slide: Conclusion
	\begin{frame}
		\frametitle{Conclusion}
		\begin{itemize}
			\item Graph Theory provides essential tools for modeling and solving problems in various fields.
			\item Understanding basic definitions and properties is crucial for progress in advanced topics.
		\end{itemize}
	\end{frame}
	
\end{document}
```

This LaTeX file presents the fundamentals of Graph Theory in a structured way, with slides covering the introduction, basic definitions, types of graphs, graph representations, fundamental theorems, applications, and a conclusion.

Please make sure to have appropriate images (`graph.png` and `adjacency_matrix.png`) in the same directory or update the paths to your local images that will illustrate the concepts.